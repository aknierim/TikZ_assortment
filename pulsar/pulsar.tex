\documentclass{standalone}

\usepackage{fontspec}
\defaultfontfeatures{Ligatures=TeX}  % -- becomes en-dash etc.
\setmonofont{Fira Mono}[Scale=MatchLowercase]

\usepackage{tikz}
\usetikzlibrary{
  calc,
  patterns,
  quotes,
  tikzmark,
  angles,
  decorations.markings,
  decorations.pathmorphing,
  decorations.pathreplacing,
  fadings,
  shapes,
  positioning
}
\tikzset{fontscale/.style = {font=\relsize{#1}}}
\usepackage{pgfplots}
\usepackage{tikz-feynman}

% math
\usepackage{amsmath}
\usepackage{amssymb}
\usepackage{mathtools}
\usepackage{xfrac}


\usepackage[
  mathrm=sym,        % use math font for mathrm, not text font
  math-style=ISO,
  bold-style=ISO,
  sans-style=italic,
  nabla=upright,
  partial=upright,
  warnings-off={
    mathtools-colon,
    mathtools-overbracket,
  },
]{unicode-math}

\usepackage[
  separate-uncertainty=true,
  per-mode=symbol-or-fraction,
]{siunitx}
\sisetup{math-micro=\text{µ},text-micro=µ}

\usepackage{mhchem}

% new commands
\newcommand{\code}[2]{%
  \texttt{\textcolor{#1}{\detokenize{#2}}}%
}


\newcommand{\ubeta}{\symup{\beta}}
\newcommand{\ugamma}{\symup{\gamma}}


\usepackage{xcolor}

% define your colors here
\xdefinecolor{tugreen}{RGB}{132, 184, 25}
\xdefinecolor{tulightgreen}{HTML}{99b560}
\xdefinecolor{darkmode}{HTML}{3a3d41}
\colorlet{tulight}{tugreen!20!white}
\colorlet{tudark}{tugreen!80!black}

\xdefinecolor{tuorange}{RGB}{227, 105, 19}
\xdefinecolor{tuyellow}{RGB}{242, 189, 0}
\xdefinecolor{tucitron}{RGB}{249, 219, 0}

\xdefinecolor{tublue}{RGB}{25, 132, 184}
\colorlet{tublight}{tublue!20!white}
\colorlet{tubdark}{tublue!60!black}

\colorlet{lightgray}{darkgray!70!white}
\colorlet{lightergray}{darkgray!50!white}


\setmainfont{Fira Sans}

\xdefinecolor{jetcolor}{HTML}{ebc934}
\xdefinecolor{magnaxis}{HTML}{eb4634}
\xdefinecolor{neutronblue}{HTML}{3486eb}


%% Modified from the dipolar magnetic field drawing ---
%% by Cyril Langlois
\newcommand{\fieldline}[2]{%
  {(pow(#1, 2)) * (3 * cos(#2) + cos(3 * #2))},
  {2 * (pow(#1, 2)) * (sin(#2) + sin(3 * #2))}%
}
\NewDocumentCommand\pgfmathsinandcos {m m m}{%
  \pgfmathsetmacro#1{sin(#3)}%
  \pgfmathsetmacro#2{cos(#3)}%
}
\NewDocumentCommand\LongitudePlane {m m m} {%
  \pgfmathsinandcos\sinel\cosel{#2} % elevation
  \pgfmathsinandcos\sint\cost{#3}   % azimuth
  \tikzset{%
    #1/.estyle={%
      cm={\cost, \sint*\sinel, 0, \cosel, (0, 0)}%
    }
  }
}
%% ---------------------------------------------------


\begin{document}
  \begin{tikzpicture}[
      >=latex
    ]
    \newdimen\R  % Radius
    \R=0.5cm

    \def\angEl{30}  % elevation angle

    %% coordinates ------------------------------------
    \coordinate (O) at (0,0);

    % Northern jet
    \coordinate (jet_N1) at (60:\R + 0.25cm);
    \coordinate (jet_N2) at (70:\R + 0.25cm);
    \coordinate (jet_N3) at (60:\R + 3cm);
    \coordinate (jet_N4) at (70:\R + 3cm);
    \coordinate (jet_eN1) at (65:\R + 0.25cm);
    \coordinate (jet_eN2) at (65:\R + 3cm);

    % Northern jet
    \coordinate (jet_S1) at (240:\R + 0.25cm);
    \coordinate (jet_S2) at (250:\R + 0.25cm);
    \coordinate (jet_S3) at (240:\R + 3cm);
    \coordinate (jet_S4) at (250:\R + 3cm);
    \coordinate (jet_eS1) at (245:\R + 0.25cm);
    \coordinate (jet_eS2) at (245:\R + 3cm);

    % Anchor points
    \coordinate (N) at (0, 3.8);
    \coordinate (S) at (0, -3.8);
    \coordinate (NE) at (1.7, 3.8);
    \coordinate (NW) at (-1.7, 3.8);
    \coordinate (SE) at (1.7, -3.8);
    \coordinate (SW) at (-1.7, -3.8);
    %% ------------------------------------------------


    %% Light cylinder
    \draw (S) ellipse (1.7cm and 0.3cm);
    \draw (N) ellipse (1.7cm and 0.3cm);
    \draw [dashed, opacity=0.7] (NE) -- (SE);
    \draw [dashed, opacity=0.7] (NW) -- (SW);


    %% Field lines -----------------------------------
    %% Adapted from Cyril Langlois' drawing
    \begin{scope}[
      rotate around={-25:(O)},
      field line/.style={
        color=gray,
        smooth,
        opacity=0.7,
        variable=\t, samples at={0,-5,-10,...,-360}
      }
    ]
      \clip [rotate around={25:(O)}] (-2.5,-4.5) rectangle (2.5,4.5);

      \LongitudePlane{{{0}zplane}}{\angEl}{0}
      \foreach \r in {0.45,0.55,0.65,1,1.2,1.7} {
        \draw[{{0}zplane}, field line] plot (\fieldline{\r}{\t});
      }
    \end{scope}
    %% ------------------------------------------------


    %% Magnetic axis
    \draw [color=magnaxis] (O) -- (jet_eN1);


    %% Northern jet -----------------------------------
    \draw [jetcolor] (jet_N1) -- (jet_N3);
    \draw [jetcolor] (jet_N2) -- (jet_N4);
    \fill [
      jetcolor,
      rotate around={-25:(jet_eN1)},
    ] (jet_eN1) ellipse (0.073cm and 0.025cm);

    % Magnetic axis
    \draw [color=magnaxis] (jet_eN1) -- (jet_eN2);

    \fill [
      jetcolor,
      rotate around={-25:(jet_eN2)},
    ] (jet_eN2) ellipse (0.314cm and 0.1cm);

    % Magnetic axis
    \draw [color=magnaxis] (jet_eN2) -- (65:\R + 4cm);
    %% ------------------------------------------------


    %% Southern jet ----------------------------------
    \draw [jetcolor] (jet_S1) -- (jet_S3);
    \draw [jetcolor] (jet_S2) -- (jet_S4);

    % Magnetic axis
    \draw [color=magnaxis] (jet_eS2) -- (245:\R + 4cm);
    \fill [
      jetcolor,
      rotate around={-25:(jet_eS2)},
    ] (jet_eS2) ellipse (0.314cm and 0.1cm);

    % Magnetic axis
    \draw [color=magnaxis] (jet_eS1) -- (jet_eS2);
    \fill [
      jetcolor,
      rotate around={-25:(jet_eS1)},
    ] (jet_eS1) ellipse (0.073cm and 0.025cm);

    \draw [color=magnaxis] (O) -- (jet_eS1);
    %% ------------------------------------------------


    \draw [neutronblue!70!black,->] (-0.75, 3) arc [
      start angle=-180,
      end angle=300,
      x radius=0.75cm,
      y radius=0.1cm
    ];

    %% Rotation axis
    \draw [neutronblue!70!black] (S) -- (N);

    %% Neutron star
    \fill [neutronblue] (O) circle (\R) coordinate (O);

    %% Annotations
    \node [
      white,
      align=center,
      font=\tiny
    ] at (O) {Neutron\\star};
    \node [
      neutronblue!70!black,
      anchor=east,
      font=\tiny
    ] at (-0.2,2.8) {Rotation axis};
  \end{tikzpicture}
\end{document}
